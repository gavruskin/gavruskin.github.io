\documentclass{beamer}

\synctex=1

\mode<presentation>
{
  \usetheme{Warsaw}
  \setbeamercovered{invisible}
}

\setbeamertemplate{navigation symbols}{}
\setbeamertemplate{footline}[frame number]{}

\graphicspath{ {./images/} }

\usepackage[british]{babel}
\usepackage{times}
\usepackage[latin1]{inputenc}
\usepackage[T1]{fontenc}
\usepackage{lmodern}
\usepackage{amsmath,amsthm,amssymb}
\usepackage{hyperref}
\usepackage{nccfoots}

\renewcommand\thempfootnote{*}

\renewcommand{\t}{$\mathrm{t}$}
\newcommand{\BHV}{$\mathrm{BHV}$}
\newcommand{\NP}{$\mathrm{NP}$}
\newcommand{\MCMC}{$\mathrm{MCMC}$}
\newcommand{\SPR}{$\mathrm{SPR}$}

\newtheorem{oobservation}{Trivial observation}

\title[AWC Meeting: {\em Phylogenetic Markov chains}]{Convergence and curvature of phylogenetic Markov chains}
\author{Alex Gavryushkin}
\titlegraphic{\includegraphics[height=2cm]{UniAuckland}}

\date{21st October 2015}

\begin{document}

\begin{frame}[plain]
\titlepage
\end{frame}

\begin{frame}{Ricci-Ollivier curvature}
\begin{definition}[Ollivier~{[2009]}]
Let $({\cal T},d)$ be a metric (tree) space with a random walk
\vskip-3mm
\[
m = (m_T)_{T\in{\cal T}}.
\]
Let $T,R \in {\cal T}$ be two distinct points (trees).
The Ricci-Ollivier curvature of $({\cal T},d,m)$ along $\overrightarrow{TR}$ is
\vskip-3mm
\[
\kappa_m(T,R) = 1 - \frac{W(m_T ,m_R)}{d(T,R)},
\]
where $W(\cdot\,,\cdot)$ is the earth mover's distance.
\end{definition}
\end{frame}

\begin{frame}{Curvature of Markov chains on graphs}

\begin{theorem}[Ollivier~{[2009]}]
If $({\cal T}, d)$ is a geodesic space then curvature is a local property.
\end{theorem}

\begin{definition}
Let $({\cal T}, d)$ be a graph with a Markov chain $m$.Then the {\em curvature of the Markov chain} $m$ on the graph $\cal T$ is the greatest number $\chi_m$ such that
\vskip-5mm
\[
\chi_m \leq \kappa_m(T,R) \mbox{ for adjacent $T$ and $R$}.
\]
\end{definition}

\begin{oobservation}
Under a distance-one random walk, the following is true for any finite metric $d$ and any pair of points $x,y$:
\vskip-3mm
\[
\dfrac{-2}{d(x,y)} \leq \kappa(x,y) \leq \dfrac{2}{d(x,y)}.
\]
\end{oobservation}
\end{frame}

\begin{frame}{Random walks}
For now, we consider three simplest random walks on various phylogenetic tree spaces.

\begin{itemize}
\item Metropolis-Hastings random walk: Choose a tree from the one neighbourhood and accept it with probability $\min(1, \dfrac{|N_1(T_{old})|}{|N_1(T_{new})|})$.
\item Uniform random walk.
\item Uniform $p$-lazy random walk, where $p$ is the laziness probability.
\end{itemize}
\end{frame}

\begin{frame}{}
\begin{theorem}[Whidden and Matsen {[2015]}]
\textup{(1)} The curvature of the uniform random walk on the SPR graph on rooted trees with $n$ leaves is bounded from below by
\vskip-3mm
	$$\frac{-n^2 + 2n}{3.5n^2 - 15n + 16} \geq -2/5$$

\textup{(2)} Subtract $1/6$ to get a lower bound on the curvature of the Metropolis-Hastings random walk.
\end{theorem}

\begin{theorem}[G, Whidden, and Matsen {[2015]}]
\textup{(1)} The curvature of the $p$-lazy uniform random walk on the SPR graph, the NNI graph, the $\tau$-graph, and the discrete $\tau$-space on rooted trees with $n$ leaves is bounded from below by
\vskip-3mm
\[
-p\,\frac{n-3}{n-2}~\cdot
\]
\textup{(2)} Subtract $2/3$ to get a lower bound on the curvature of the Metropolis-Hastings random walk.
\end{theorem}
\end{frame}

\begin{frame}{}
\begin{theorem}[Whidden and Matsen {[2015]}]
The maximum curvature of the uniform random walk on the SPR graph between two adjacent trees with $n$ leaves is
\vskip-3mm
$$\frac{6n-17}{3n^2-13n+14}.$$
\end{theorem}

\begin{theorem}[G, Whidden, and Matsen {[2015]}]
The curvature of a uniform random walk on the discrete $\tau$-space satisfies
\vskip-5mm
\[
\kappa_{d\tau}(T,R) \leq
\dfrac{1}{2(n-2)}.
\]
\end{theorem}
\end{frame}

\begin{frame}{}

\begin{theorem}[G, Whidden, and Matsen {[2015]}]
Let $\{T_n \mid n \in \mathbb N\}$ and $\{S_n \mid n \in \mathbb N\}$ be two sequences of phylogenetic trees such that $d(T_n,R_n) = 1$ for all $n$.
Then $$\lim_{n \rightarrow \infty} \kappa_n(T_n,S_n) = 0$$ for the uniform random walk on the SPR
graph\footnote{For the SPR graph, we have to bound the size of the subtree which is getting moved.},
the NNI graph, the $\tau$-graph, and the discrete $\tau$-space.
\end{theorem}
\end{frame}

\begin{frame}{\href{http://alex.gavruskin.com/pictures/}{\Large{Thank
you for your attention!}}}
\thebibliography{9}
\bibliographystyle{alpha}

\scriptsize

\bibitem{Ollivier}
Yann Ollivier.
\newblock Ricci curvature of Markov chains on metric spaces.
\newblock {\em J.\ Functional Analysis,} 256, 3, 810--864, 2009.

\bibitem{Gavruskin2015}
Alex Gavryushkin and Alexei Drummond.
\newblock The space of ultrametric phylogenetic trees.
\newblock {\em arXiv preprint} \href{http://arxiv.org/abs/1410.3544}{arXiv:1410.3544}, 2014.

\bibitem{chrisErick}
Chris Whidden and Frederick A.\ Matsen IV.
\newblock Quantifying MCMC exploration of phylogenetic tree space.
\newblock {\em Systematic Biology}, doi:10.1093/sysbio/syv006, 2015.

\bibitem{chrisErick2015}
Chris Whidden and Frederick A.\ Matsen IV.
\newblock Ricci-Ollivier curvature of two random walks on rooted phylogenetic subtree-prune-regraft graph.
\newblock To appear in the proceedings of the {\em Thirteenth Workshop on Analytic Algorithmics and Combinatorics,} 2015.

\bibitem{chrisErickG2015}
Alex Gavryushkin, Chris Whidden, and Frederick A.\ Matsen IV.
\newblock Random walks over discrete time-trees.
\newblock To appear on the {\em arXiv,} 2015.

\end{frame}
\end{document}
